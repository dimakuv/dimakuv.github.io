% LaTeX Curriculum Vitae Template
%
% Copyright (C) 2004-2009 Jason Blevins <jrblevin@sdf.lonestar.org>
% http://jblevins.org/projects/cv-template/
%
% You may use use this document as a template to create your own CV
% and you may redistribute the source code freely. No attribution is
% required in any resulting documents. I do ask that you please leave
% this notice and the above URL in the source code if you choose to
% redistribute this file.

\documentclass[letterpaper]{article}

\usepackage{hyperref}
\usepackage{geometry}
\usepackage{enumitem}
% Comment the following lines to use the default Computer Modern font
% instead of the Palatino font provided by the mathpazo package.
% Remove the 'osf' bit if you don't like the old style figures.
\usepackage[T1]{fontenc}
\usepackage[sc,osf]{mathpazo}
\usepackage{color}


% Set your name here
\def\name{Dmitrii Kuvaiskii}

% Replace this with a link to your CV if you like, or set it empty
% (as in \def\footerlink{}) to remove the link in the footer:
\def\footerlink{}

% The following metadata will show up in the PDF properties
\hypersetup{
  colorlinks = true,
  urlcolor = blue,
  pdfauthor = {\name},
  pdfkeywords = {computer science, systems engineering},
  pdftitle = {\name: Curriculum Vitae},
  pdfsubject = {Curriculum Vitae},
  pdfpagemode = UseNone
}


\geometry{
  body={6.5in, 8.5in},
  left=1.0in,
  top=1.25in
}

% Customize page headers
\pagestyle{myheadings}
\markright{\name}
\thispagestyle{empty}

% Custom section fonts
\usepackage{sectsty}
\sectionfont{\rmfamily\mdseries\Large}
\subsectionfont{\rmfamily\mdseries\itshape\large}


% Don't indent paragraphs.
\setlength\parindent{0em}

% Make lists without bullets
\renewenvironment{itemize}{
  \begin{list}{}{
    \setlength{\leftmargin}{1.5em}
  }
}{
  \end{list}
}

\begin{document}

% Place name at left
{\huge \name}

% Alternatively, print name centered and bold:
%\centerline{\huge \bf \name}

\vspace{0.25in}

\begin{minipage}{0.45\linewidth}
Address:\\
Noethnitzer strasse 25\\
%Technical University of Dresden,\\
01187 Dresden\\
Germany\\
%  \href{http://www.mpi-sws.org}{Max Planck Institute for Software Systems} \\

 
\end{minipage}
\begin{minipage}{0.45\linewidth}
  \begin{tabular}{ll}
      Mobile: & +49 (176) 9978 5337\\
      Email: & \href{mailto:dmitrii.kuvaiskii@gmail.com}{\tt dmitrii.kuvaiskii@gmail.com} \\
    Homepage: & \href{https://dimakuv.github.io/}{\tt https://dimakuv.github.io/} \\
    GitHub: & \href{https://github.com/dimakuv}{\tt https://github.com/dimakuv} \\
%    Birthdate: & 28th November, 1987
  \end{tabular}
\end{minipage}


\section*{Research Interests}

My research interests lie in the field of dependability in software systems, with a particular focus on fault tolerance and security. %Within these fields, I investigate the applicability of modern hardware extensions to increase reliability of real-world applications while imposing low overheads.

%My research interests revolve around dependability in software systems, with the focus on software-based fault tolerance and security for legacy C/C++ programs.
%In particular, I leverage recent sets of extensions in Intel processors (AVX, TSX, MPX, SGX) to decrease the performance penalty induced by current techniques.
%In the field of fault tolerance, I develop compiler frameworks which automatically transform applications to detect and
%tolerate transient and permanent faults occurring in hardware. In particular, I utilize the recent sets
%of extensions in Intel processors (AVX, TSX) to decrease the performance penalty induced
%by the proposed techniques.
%In the field of security, I concentrate on bounds-checking mechanisms to transparently retrofit memory safety (e.g., protection against buffer overflows) in legacy systems.
%In particular, I work with Intel MPX and SGX extensions to build bounds-checking approaches and analyze their performance overheads.


\section*{Education}

\begin{itemize}

 \item \textbf{Ph.D. Candidate} in Computer Science (Dec 2013 - present)\\
 {\em  Technische Universit{\"a}t Dresden (TU Dresden), Germany}\\
Advisors: Prof. Dr. Christof Fetzer and Prof. Dr. Pramod Bhatotia

 \item \textbf{Master of Science} in Computer Science (Oct 2011 - Nov 2013) \\
 {\em  Technische Universit{\"a}t Dresden (TU Dresden), Germany}


\item \textbf{Diplom} in Electrical Engineering (Sep 2004 - Jul 2010)\\
  {\em Bauman University, Moscow, Russia} 

\end{itemize}



\section*{Employment}

{\bf Auriga Inc, Moscow, Russia} (Sep 2010 - Aug 2011)\\
{\em Certification engineer, Software developer}
\begin{itemize}
	\item Responsibilities:
		\begin{itemize}
		\item --- documenting and testing code of the PikeOS embedded operating system (C);
		\item --- writing medical special-purpose programs (C++ and C\#).
		\end{itemize}
	\item 
\end{itemize}

{\bf Diasoft, Moscow, Russia} (Sep 2007 - Aug 2010)\\
{\em Software developer}
\begin{itemize}
	\item Responsibilities: programming insurance subsystems using Transact SQL and Delphi.
\end{itemize}


\section*{Honors and Awards}

\begin{itemize}
\item \textbf{Carter Award (Best student paper)} at DSN'15
\item \textbf{Best paper award} at SRDS'14
\item \textbf{Erasmus Mundus Action 2 MULTIC} scholarship, 2011-2013

\end{itemize}


\section* {Ph.D. Dissertation}

\begin{itemize}
\item {\bf Topic:} Dependable Systems Leveraging new ISA extensions (preliminary)\\
%\vspace{3pt}
{\bf Supervisors:} Prof. Dr. Christof Fetzer and Prof. Dr. Pramod Bhatotia\vspace{-13pt}\\

In the context of my Ph.D. dissertation, I investigate and build systems to increase software dependability leveraging recent sets of ISA extensions in Intel processors, with the focus on software-based fault tolerance and security for legacy C/C++ programs. %In particular, I leverage recent sets of extensions in Intel processors (AVX, TSX, MPX, SGX) to decrease the performance penalty induced by current dependability techniques.\\

{\bf Research projects:}
	\begin{itemize}%
	
		\item  {\bf Intel MPX Explained}: Detailed evaluation of {\tt Intel MPX} and discussion of its applicability in comparison to other bounds-checking approaches.\\
			--- Software: \href{https://intel-mpx.github.io/}{Intel-MPX.github.io}
		
		\item {\bf SGXBounds}: LLVM-based bounds checker to detect and tolerate security bugs in multithreaded legacy C/C++ programs inside {\tt Intel SGX} enclaves.\\
			--- Software: \href{https://github.com/tudinfse/sgxbounds}{https://github.com/tudinfse/sgxbounds}
			
			
		\item {\bf Elzar}: LLVM compiler pass to detect and mask transient CPU faults in multithreaded legacy C/C++ programs using {\tt Intel AVX}.\\
		 --- Software: \href{https://github.com/tudinfse/elzar}{https://github.com/tudinfse/elzar}
		 
		\item {\bf HAFT}: LLVM compiler pass to detect and tolerate transient CPU faults in multithreaded legacy C/C++ programs using {\tt Intel TSX}.\\
				--- Software: \href{https://github.com/tudinfse/haft}{https://github.com/tudinfse/haft}
				
		\item {\bf $\Delta$-Encoding}: Source-to-source compiler to detect transient and permanent CPU faults in legacy C programs utilizing unused IPC resources of modern CPUs.
	\end{itemize}%
\end{itemize}%	


\section*{Publications}

{\bf Conference publications:}
%\begin{itemize}
\begin{enumerate} [label= $\lbrack$\arabic*$\rbrack$, resume]
\item Intel MPX Explained\\
{\em Oleksii Oleksenko, {Dmitrii Kuvaiskii}, Pramod Bhatotia,  Pascal Felber, and Christof Fetzer}\\
{\bf USENIX ATC 2017} {\em (Under submission)}

\item SGXBounds: Memory Safety for Shielded Execution\\
{\em Dmitrii Kuvaiskii, Oleksii Oleksenko, Sergei Arnautov, Bohdan Trach, Pramod Bhatotia, Pascal Felber, and Christof Fetzer}\\
{\bf  EuroSys 2017} {\em (Under submission)}

\item Elzar: Triple Modular Redundancy using Intel Advanced Vector Extensions\\
{\em Dmitrii Kuvaiskii, Oleksii Oleksenko, Pramod Bhatotia, Pascal Felber, and  Christof Fetzer}\\
{\bf  DSN 2016}


\item HAFT: Hardware-Assisted Fault Tolerance\\
{\em Dmitrii Kuvaiskii, Rasha Faqeh, Pramod Bhatotia, Pascal Felber, and  Christof Fetzer}\\
{\bf  EuroSys 2016}


\item $\Delta$-Encoding: Practical Encoded Processing\\
{\em Dmitrii Kuvaiskii and Christof Fetzer}\\
{\bf   DSN 2015} \underline{Carter Award (Best student paper)}

%\item Needles in the Haystack---Tackling Bit Flips in Lightweight Compressed Data\\
%{\em Till Kolditz, Dirk Habich, {Dmitrii Kuvaiskii}, Wolfgang Lehner, and Christof Fetzer}\\
%{\bf DATA 2015}.

\item HardPaxos: Replication hardened against hardware errors\\
{\em Diogo Behrens, {Dmitrii Kuvaiskii}, and Christof Fetzer}\\
{\bf SRDS 2014} \underline{Best paper award}

\end{enumerate}



{\bf Extended abstracts:}

\begin{enumerate} [label= $\lbrack$\arabic*$\rbrack$, resume]

\item{Efficient Fault Tolerance using Intel MPX and TSX}\\
{\em Oleksii Oleksenko, {Dmitrii Kuvaiskii}, Pramod Bhatotia, Christof Fetzer, and Pascal Felber}\\
Fast abstract at {\bf DSN 2016}

%\item{Practical Encoded Processing}\\
%{\em {Dmitrii Kuvaiskii} and Christof Fetzer}\\
%Poster at {\bf SRDS 2014}.

\end{enumerate}


\section*{Talks}
\begin{itemize}
\item ACM EuroSys'16, London, April 2016\\
	{\em  HAFT: Hardware-Assisted Fault Tolerance}
\item IEEE DSN'16, Toulouse, June 2016\\
	{\em  Elzar: Triple Modular Redundancy using Intel Advanced Vector Extensions}
\item IEEE DSN'15, Rio de Janeiro, June 2015\\
{\em  $\Delta$-Encoding: Practical Encoded Processing}
\end{itemize}


\section*{Teaching Experience}

\begin{itemize}
\item {\bf Teaching assistant:} Distributed Systems Engineering (DSE) courses, TU Dresden, Dec 2013 - present.
\begin{itemize}
	\item --- Concurrent and Distributed Systems lab, summer semesters: 2014, 2015, \& 2016
	\item --- Principles of Dependable Systems exercises, winter semesters: 2014, 2015, \& 2016
	\item --- Software Fault Tolerance exercises, summer semesters: 2014, 2015, \& 2016
\end{itemize}
\end{itemize}


\section*{Professional Activities}


\begin{itemize}
 \item { Shadow PC member:} {\bf EuroSys 2016}
\end{itemize}
  

\section*{Skills}
\begin{itemize}
	\item {\bf Languages}: C, C++, Assembly (expert), Unix shell, Python, R (competent)
	\item {\bf Frameworks}: LLVM, gdb, Intel Pin, Intel SDE
	\item {\bf Technologies}: Intel SSE/AVX, Intel TSX, Intel MPX, Intel SGX
\end{itemize}


\section*{References}

\begin{itemize}

\item {\bf Prof. Dr. Christof Fetzer }  \\
TU Dresden, Germany\\
Email: \href{mailto:christof.fetzer@tu-dresden.de}{\tt christof.fetzer@tu-dresden.de}
      
\item {\bf  Prof. Dr. Pramod Bhatotia }  \\
University of Edinburgh, UK\\
Email: \href{mailto:pramod.bhatotia@gmail.com}{\tt pramod.bhatotia@gmail.com}

\item {\bf Prof. Dr. Pascal Felber} \\
University of Neuchatel, Switzerland\\
Email: \href{mailto:pascal.felber@unine.ch}{\tt pascal.felber@unine.ch}

\end{itemize}


\end{document}
